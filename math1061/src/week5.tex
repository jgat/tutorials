\documentclass[a4paper,12pt]{article}
\usepackage{amsmath}
\usepackage{amssymb}
\usepackage{amsthm}

\newcommand{\floor}[1]{\left\lfloor #1 \right\rfloor}
\newcommand{\ceil}[1]{\left\lceil #1 \right\rceil}
\begin{document}

\pagestyle{myheadings}
\markright{MATH1061 | Week 5 Tutorial (T5)}

\title{MATH1061 | Week 5 Tutorial (T5)}
\date{26 March 2013}
\maketitle

\begin{enumerate}
\item Prove that there is no integer which is both even and odd.
\item Consider the statement ``$\forall n \in \mathbb{Z}, k \in \mathbb{Z}^+$,
      if $5 \nmid n^k$, then $5 \nmid n$.''
\begin{itemize}
    \item[(a)] Prove this statement using a proof by contraposition.
    \item[(b)] Prove this statement using a proof by contradiction.
\end{itemize}
\item For all positive integers $a, b$, prove that $a \mid b$ if and only if
$\gcd(a, b) = a$. (Note that to prove ``X if and only if Y'', you need to prove
both ``if X, then Y'' and ``if Y, then X'';
because $p \leftrightarrow q \equiv (p \to q) \land (q \to p)$.)
%\item Define $\gcd(a, b, c)$ to be the largest integer which is a common divisor
%of $a$, $b$, and $c$. It is true that $\gcd(a, b, c) = \gcd(a, \gcd(b, c))$.
%Use this fact to compute $\gcd(429, 273, 231)$.
\item Find all possible values of $196x + 158y$ for integers $x$ and $y$.
\item Find a solution to the equation $64x + 28y = 12$ where $x, y$ are
integers. What is another solution?
\end{enumerate}

\newpage

Solutions:

\begin{enumerate}
\item Proof by contradiction: suppose that there is an integer $n$ which is
both even and odd. Then, $n = 2a = 2b + 1$ for some integers $a$ and $b$. Then,
$a = b + \frac{1}{2}$, which is not an integer, a contradiction. So, our
assumption is incorrect and the original claim is true.

\item (a) Let $n$ be an integer and $k$ be a positive integer.
Suppose $5 \mid n$, i.e. $n = 5x$ for some $x \in \mathbb{Z}$. Then,
$n^k = 5^k x^k = 5(5^{k-1}x^k)$, and since $5^{k-1}x^k \in \mathbb{Z}$, $5 \mid n^k$.
\\
(b) Suppose there is an integer $n$ and $k$ with $5 \nmid n^k$ and $5 \mid n$.
Then, $n = 5x$ for $x \in \mathbb{Z}$, so $n^k = 5(5^{k-1}x^k)$, and since
$5^{k-1}x^k \in \mathbb{Z}$, $5 \mid n^k$, which contradicts the fact that
$5 \nmid n^k$, so our assumption is incorrect, and the original claim is true.

\item Let $a, b$ be positive integers. First, we prove that ``if $a \mid b$,
then $\gcd(a, b) = a$'': Suppose $a \mid b$. Then, since $a \mid a$, $a$ is a
common divisor of $a$ and $b$. If $c$ is a common divisor of $a$ and $b$, then
$c \mid a$, which means that $c \leq a$, so $a$ is the greatest common divisor,
of $a$ and $b$.
\\
Next, we prove that ``if $\gcd(a, b) = a$, then $a \mid b$'': Suppose that
$\gcd(a, b) = a$. Then, $a$ is a common divisor of $a$ and $b$; in particular,
$a \mid b$.

\item Using the theorem on page 81 of the workbook, we see that all possible
values of $196x + 158y$ are the integers $c$ such that $\gcd(196, 158) \mid c$.
By the Euclidean algorithm,
$\gcd(196, 158) = \gcd(158, 38) = \gcd(38, 6) = \gcd(6, 2) = \gcd(2, 0) = 2$, so
$196x + 158y = c$ if and only if $2 \mid c$.
So, the even integers are all the possible values of $196x + 158y$.

\item By the Extended Euclidean algorithm, we find that $\gcd(64, 28) = 4$, and
$4 = 64 \cdot -3 + 28 \cdot 7$. Multiplying by $3$, we have
$12 = 64 \cdot -9 + 28 \cdot 7$ (*), so one solution is $(x, y) = (-9, 7)$.
Since $0 = 64 \cdot 28 + 28 \cdot -64$, we can add this on to (*) to obtain
$12 = 64 \cdot 19 + 28 \cdot -43$, so another solution is
$(x, y) = (19, -43)$.
\end{enumerate}

\end{document}
