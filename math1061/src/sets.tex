\documentclass[a4paper,12pt]{article}
\usepackage[lmargin=2.5cm,rmargin=2.5cm,tmargin=2cm,bmargin=2cm]{geometry}
\usepackage{amsmath}
\usepackage{amssymb}

\begin{document}

\pagestyle{myheadings}
\markright{MATH1061 | Set Theory Tutorial}

\begin{enumerate}
\item Prove that for arbitrary sets $A,B,C$ that
\[
    (A \setminus B) \setminus C = A \setminus (B \cup C).
\]
\item Prove that for arbitrary sets $A$ and $B$ we have $A \subseteq B$ if and
only if $A \cup B = B$.
\item Let $X = \{1, \{1\}, \{\emptyset\}\}$.
Write down the elements of $\mathcal{P}(X) \setminus X$.
\end{enumerate}

Challenge: If $X$ is a finite set, prove by induction that
$|\mathcal{P}(X)| = 2^{|X|}$.

Challenge: Prove that for arbitrary sets $A$ and $B$, that $A \subseteq B$ iff
$\mathcal{P}(A) \subseteq \mathcal{P}(B)$.
\\

Solution/outlines:

\begin{enumerate}
\item Element argument. $x \in (A \setminus B) \setminus C)$ iff
$x \in A$ and $x \not \in B$ and $x \not \in C$, iff $x \in A$ and
$\sim((x \in B) \lor (x \in C))$, iff $x \in A \setminus (B \cup C)$.
\item Construct a correspondence with the predicate
\[
    (P(x) \to Q(x)) \leftrightarrow ((P(x) \lor Q(x)) \leftrightarrow Q(x)),
\]
show that this is a tautology (truth table or otherwise). Or, show that if
$A \subseteq B$ then $A \cup B \subseteq B \subseteq A \cup B$, and if
$A \cup B = B$ then $A \subseteq B$.
\item $\mathcal{P}(X) \setminus X = \big\{\emptyset, \{\{1\}\}, \{\{\emptyset\}\},
\{1, \{1\}\}, \{1, \{\emptyset\}\}, \{\{1\}, \{\emptyset\}\},
\{1, \{1\}, \{\emptyset\}\}\big\}$.
\item (Challenge) Let $n = |X|$, induct on $n$. Base case: $\mathcal{P}(\emptyset)
= \{\emptyset\}$. Inductive step: Pick $x \in X$, $X' = X \setminus \{x\}$,
\[
    \mathcal{P}(X) = \mathcal{P}(X) \cup \{ A \cup \{x\} \mid A \in \mathcal{P}(X)  \}.
\]
\item (Challenge) ($\Rightarrow$) If $A \subseteq B$, then for any $X \in
\mathcal{P}(A)$, we have $X \subseteq A \subseteq B$, so $X \subseteq B$, so
$X \in \mathcal{P}(B)$. Therefore, $\mathcal{P}(A) \subseteq \mathcal{P}(B)$.

($\Leftarrow$) If $\mathcal{P}(A) \subseteq \mathcal{P}(B)$, then in particular,
$A \in \mathcal{P}(A) \subseteq \mathcal{P}(B)$, so $A \subseteq B$.
\end{enumerate}

\end{document}
