\documentclass[a4paper,12pt]{article}
\usepackage[lmargin=2.5cm,rmargin=2.5cm,tmargin=2cm,bmargin=2cm]{geometry}
\usepackage{amsmath}
\usepackage{amssymb}
\usepackage{amsthm}
\usepackage{algpseudocode}

\newtheorem{theorem}{Theorem}
\begin{document}

\pagestyle{myheadings}
\markright{MATH1061 | Week 6 Tutorial (T5)}

%\title{MATH1061 | Week 6 Tutorial (T5)}
%\date{9 April 2013}
%\maketitle

Related reading: Workbook chapters D.1 through D.6.

{\bf Exercises:}

\begin{enumerate}
\item
For a positive integer $n$, use $\sum$ notation to write an expression for the
first $n$ terms of the sum
\[
    2 + 6 + 18 + 54 + 162 + \dots
\]
Show that the first $n$ terms of this sum is equal to $3^n-1$.
\item
For what positive integers $n$ is it true that $n! > 2^n$? Use induction to
prove that it is true for these values of $n$.
\item
The {\em Fibonacci sequence} is the sequence $\{F_n\}$ defined by
\[
    F_0 = 0; \quad F_1 = 1; \quad \forall n \geq 2,\; F_n = F_{n-1} + F_{n-2}.
\]
Let $\phi = \frac{1 + \sqrt{5}}{2}$. It can be shown that $\phi \approx 1.618$
and that $\phi^2 = \phi + 1$. Use these facts to prove that $F_n < \phi^n$ for
all $n \geq 0$.
\item
\footnote{This question is from Donald Knuth's The Art of Computer
Programming (3rd ed.), \S 1.2.1, Q.2}
There must be something wrong with the following proof. What is it?
\begin{theorem}
Let $a$ be any positive number. For all integers $n \geq 0$, we have $a^n = 1$.
\end{theorem}
\begin{proof}
If $n = 0$, $a^n = a^0 = 1$, so the base case holds.
If the theorem is true for all of $0, 1, \dots, n-1, n$ (i.e. $a^0 = 1, a^1 = 1,
\dots, a^{n-1}=1, a^n = 1$), then we have
\[
    a^{n+1} = \frac{a^n \cdot a^n}{a^{n-1}} = \frac{1 \cdot 1}{1} = 1,
\]
so the theorem is true for $n+1$ as well. By the principle of strong induction,
$a^n = 1$ for all integers $n \geq 0$.
\end{proof}
\item
Define a sequence $\{a_n\}$ as follows: $a_1 = 5$, and for $n \geq 1$, $a_n =
a_{n-1} + 2n + 1$.
Guess a closed form (i.e. not recursive) expression for $a_n$. Use induction to
prove the guess is correct.
\end{enumerate}

{\bf Additional exercises:}

\begin{enumerate}
\item[6.] Show that
\[
    \sum_{k=1}^n 3 \cdot 4^{k-1} = 4^n - 1.
\]
Compare this equation to Exercise 1 above, and Example D.2.4 (p.96 of
workbook), and suggest a generalisation. Prove this generalisation is true.
Use this to factorise the polynomial $x^n - 1$.
\item[7.] Prove that the sum of the cubes of the first $n$ positive integers is
equal to the square of the sum of the first $n$ positive integers. That is,
for all integers $n \geq 1$, prove that
\[ 1^3 + 2^3 + 3^3 + \dots + n^3 = (1 + 2 + \dots + n)^2. \]
(Hint: Use the result from Example D.2.1)
\end{enumerate}

%{\bf Hints:}
%\begin{enumerate}
%\item
%Here's a pattern in the first few elements:
%$2 = 2 \cdot 3^0$, $6 = 2 \cdot 3^1$, $18 = 2 \cdot 3^2$, $54 = 2 \cdot 3^3$.
%Use an induction proof, and remember that $3^n + 2 \cdot 3^n = 3^{n+1}$.
%\item
%Check what $n!$ and $2^n$ are for small integers $n$. Use a proof by induction
%to show that it's true for all $n$ above a certain point.
%\item
%Use strong induction, and remember that
%$\phi^n + \phi^{n-1} = \phi^{n-1} (\phi + 1)$.
%\item
%What values of $n$ does the inductive step work on?
%\item
%Find the value of $a_2$, $a_3$, and $a_4$.
%\end{enumerate}

\newpage

{\bf Challenge exercises:}
(i.e., if you can do any of these questions, then you're going {\em way} above
the course expectations. But hey, maybe you like this stuff, or maybe you want
to see some applications to computer science.)

\begin{enumerate}
\item[C1.]
The following algorithm takes a non-negative integer $n$ as a parameter. Prove
that the algorithm returns the value of $n^2$. (Related reading: Epp, 4th ed.
\S 5.5)
\begin{algorithmic}[1]
\Function{square}{$n$}
    \Require $n \in \mathbb{Z}$, $n \geq 0$.
    \Ensure Return value is $n^2$.
    %\If{$n = 0$}
    %    \State \Return 0
    %\ElsIf{$n > 0$}
    %    \State \Return {\footnotesize SQUARE}($n-1$) + $2n - 1$
    %\Else
    %    \State \Return {\footnotesize SQUARE}($-n$)
    %\EndIf
    \State $m \gets 0$
    \State $k \gets 0$
    \While{$k < n$}
        \State $m \gets m + 2k + 1$
        \State $k \gets k + 1$
    \EndWhile
    \State \Return $m$
\EndFunction
\end{algorithmic}
\item[C2.]
Let $d$ be a positive integer. For any integer $n \geq 0$, prove that there
exist integers $q, r$ with $0 \leq r < d$ such that $n = dq+r$.

Hint: Use the well-ordering principle in a proof by contradiction.

Remark: This is a part of a familiar theorem, which is stated on p.58 of the
workbook.
\item[C3.]
Prove the result in Exercise C2 also holds for negative $n$.
(Hint: if $n$ is a negative integer, then $-n$ is a positive integer, and we've
already proved it for the positive integers.)
Why couldn't we easily use the well-ordering principle to prove it for all
integers $n$, instead of just the non-negative ones?
%\begin{proof}
%Let $n$ be a negative integer and $d$ be a positive integer. Then $-n \geq 0$,
%so there are integers $q, r$ such that $-n = dq + r$ and $0 \leq r < d$.
%If $r = 0$, let $q'=-q$ and $r'=0$. Then, $q'$ and $r'$ are integers such that
%$n = dq' + r'$ and $0 \leq r' < d$.
%Otherwise, $r > 0$, and let $q'=-q-1$ and $r'=d-r$. Then, $q'$ and $r'$ are
%integers such that $n = dq' + r'$ and $0 \leq r' < d$.
%\end{proof}
There's one more part of the theorem on p.58 we haven't yet proved,
what is it? Can you prove it?
\item[C4.] (Related reading: Workbook D.7)
We can define a set of arithmetic expressions over the real numbers as follows:
\begin{itemize}
\item[I.] BASE: Each real number $r$ is an arithmetic expression.
\item[II.] RECURSION: If $u$ and $v$ are arithmetic expressions, then $(u+v)$
and $(u-v)$ are arithmetic expressions.
\item[III.] RESTRICTION: There are no arithmetic expressions over the real
numbers other than those obtained from I and II.
\end{itemize}
%Show: (a) that $((0.1+(3.2-0))+(-0.2-\pi))$ is an arithmetic expression, and
%(b) that $(2 + (1.414 - 12.1) + 13.2)$ is not an arithmetic expression.
Design and implement a computer program which takes as input an arithmetic
expression as defined above, and computes the result of applying the
additions and subtractions. Extend your program to also detect for erroneous
input (i.e. input which is not an arithmetic expression).\footnote{If you're
interested in this sort of thing, consider taking COMP3506 in the future.}
\end{enumerate}


%\newpage
%\item
%Use the well-ordering principle in a proof by contradiction.
%Let $S$ be the set of all non-negative integers $n$ for which we {\em cannot}
%write $n = dq+r$ for any suitable $q, r$. Then find a contradiction
%involving the smallest element of $S$.
%\item
%(a) Show that ``3.2'', ``0'', and ``(3.2 - 0)'' are arithmetic expressions, and so on.

%(b) If it were an arithmetic expression, explain why at least one of the
%following would have to be true:
%\begin{itemize}
%\item The whole expression is a real number.
%\item ``$2+(1.414-12.1)$'' and ``$13.2$'' are both arithmetic expressions.
%\item ``$2$'' and ``$(1.414-12.1)+13.2$'' are both arithmetic expressions.
%\item ``$2+(1.414$'' and ``$12.1)+13.2$'' are both arithmetic expressions.
%\end{itemize}
%\item[C1.]
%If you can't think of a generalisation, observe that the following are also true:
%\[
%    \sum_{k=1}^n 4 \cdot 5^{k-1} = 5^n - 1; \quad \quad
%    \sum_{k=1}^n 5 \cdot 6^{k-1} = 6^n - 1.
%\]
%\item[C2.]
%Use the result from Example D.2.1, that $1 + 2 + \dots + n = \frac{n(n+1)}{2}$.
%\item[C3.]
%As an example of how to apply this algorithm, we will compute
%{\footnotesize SQUARE}(2):
%\begin{itemize}
%\item
%{\footnotesize SQUARE}(2): Since $2 > 0$, we return
%{\footnotesize SQUARE}(1) $+ 2 \cdot 2 - 1$.
%\item
%{\footnotesize SQUARE}(1): Since $1 > 0$, we return
%{\footnotesize SQUARE}(0) $+ 2 \cdot 1 - 1$.
%\item
%{\footnotesize SQUARE}(0): Since $0 = 0$, we return $0$.
%\item
%Since {\footnotesize SQUARE}(0) returned 0, {\footnotesize SQUARE}(1) will return $0 + 2 \cdot 1 - 1 = 1$.
%\item
%Since {\footnotesize SQUARE}(1) returned 1, {\footnotesize SQUARE}(2) will return $1 + 2 \cdot 2 - 1 = 4$.
%\end{itemize}
%To prove the algorithm works, let $P(n)$ be the proposition:
%{\footnotesize SQUARE}($n$) returns $n^2$. Prove $P(n)$ holds for all
%$n \geq 0$ by induction.
%\item[C4.]
%To prove it holds for the negative integers, notice that if $n$ is a negative
%integer, then $-n$ is a positive integer which we have already proved has a
%representation as $-n=dq+r$ for some $q,r$. When you pick a `new' $q'$ and $r'$
%to use for $n$, make sure that $0 \leq r' < d$.
%\item[C5.]

%No hints for the challenge tasks, otherwise they wouldn't be challenges.

\end{document}
