\documentclass[a4paper,12pt]{article}
\usepackage[lmargin=2.5cm,rmargin=2.5cm,tmargin=2cm,bmargin=2cm]{geometry}
\usepackage{amsmath}
\usepackage{amssymb}
\usepackage{amsthm}

\newtheorem{theorem}{Theorem}
\begin{document}

\pagestyle{myheadings}
\markright{MATH1061 | Week 9 Tutorial (T5)}

\title{MATH1061 | Week 9 Tutorial (T5)}
\date{30 April 2013}
\maketitle

\begin{enumerate}
\item
Do graphs with the following descriptions exist? If so, draw one, if not,
explain why.
    \begin{enumerate}
    \item A graph on 4 vertices with vertices of degree 1, 2, 3, 3.
    \item A graph on 5 vertices with vertices of degree 1, 2, 3, 3, 5.
    \item A simple graph on 6 vertices, all with degree 3.
    \item A tree on 5 vertices with no leaves.
    \item A connected graph with 8 vertices and 6 edges.
    \item A connected graph with 10 vertices, 9 edges, and at least one cycle.
    \end{enumerate}
\item
A group of 4 men and 3 women play a tennis tournament in the following way: two
people (call them $A$ and $B$) play a match, then $B$ plays with someone
else who is not $A$ (call them $C$), then $C$ plays with someone who is
not $B$, then that person plays with someone who is not $C$, and so on.
    \begin{enumerate}
    \item The group would like to play in a way that everyone plays against
       everyone else exactly once. Is it possible to do this?
    \item The group would like to play in a way that all the men play against each
       woman exactly once, all the women play against each man exactly once,
       and no two people of the same gender play against each other. Is it
       possible to do this?
    \item The group would like to play in a way that everyone plays exactly twice,
       and they only play with someone of the opposite gender. Is it possible to
       do this?
    \item Suppose now there are $n$ men and $m$ women (where $n$ and
    $m$ are positive integers). For each of the three situations above, what are
    all the values of $n$ and $m$ which make the arrangement possible?
    \end{enumerate}
\item
Challenge: Let $G$ be a graph and let $A$ be its adjacency matrix.
Make a conjecture about how the entry in row $i$ column $j$ of $A^2$
relates to the vertices $v_i$ and $v_j$ of the graph. What about the entries of
the matrix $A^n$?

(Remark: Computers are efficient at doing matrix calculations, so if we want
to write a computer program that stores and analyses graphs, we can store the
adjacency matrix of the graph in the computer).
\end{enumerate}

\newpage

{\bf Solutions:}

\begin{enumerate}
\item
    \begin{enumerate}
    \item Not possible: total degree is 9, which is odd.
    \item Possible, but not with simple graphs: An example would be to have
    a vertex of degree 5 with a loop, and a vertex of degree 3 with a loop.
    \item Possible: vertex set $\{1, 2, 3, 4, 5, 6\}$ and edge set
    $\{12, 23, 34, 45, 56, 61, 14, 26, 35\}$.
    \item Not possible: Each vertex would have degree at least 2, for a total
    degree of at least 10, but a tree on 5 vertices needs to have total degree
    8.
    \item Not possible: Using 6 edges, we can only connect at most 7 vertices.
    \item Not possible: Using 9 edges and 10 vertices, we can only construct a
    tree, but no cycles. We could construct a cycle of length $n$ and a tree
    of order $10-n$, but they would not be connected.
    \end{enumerate}
\item
    \begin{enumerate}
    \item Possible: The problem reduces to finding an Euler path or Euler
    circuit of $K_7$. Each vertex in $K_7$ has degree 6, so they're all even.
    \item Not possible: The problem reduces to finding an Euler path or Euler
    circuit of $K_{4,3}$. The vertices in the order-4 partite set have degree 3.
    \item Not possible: The problem reduces to finding a Hamiltonian circuit of
    $K_{4,3}$, and we can check that $K_{4,3}$ has no Hamiltonian circuit.
    \item
    a. is possible iff $n+m$ is odd.

    b. is possible iff $n$ and $m$ are both even (which gives an Eulerian
        circuit), or $m = 2$ and $n$ is odd, or $n = 2$ and $m$ is odd (which
        gives an Eulerian path).

    c. is possible iff $n = m$.
    \end{enumerate}
\item
The entry in row $i$ column $j$ of the matrix $A^n$ represents the number of
walks of length $n$ from $v_i$ to $v_j$.
\end{enumerate}

\end{document}
