\documentclass[a4paper,12pt]{article}
\usepackage[lmargin=2.5cm,rmargin=2.5cm,tmargin=2cm,bmargin=2cm]{geometry}
\usepackage{amsmath}
\usepackage{amssymb}
\usepackage{amsthm}

\newtheorem{theorem}{Theorem}
\begin{document}

\pagestyle{myheadings}
\markright{MATH1061 | Week 9 Tutorial (T5)}

\title{MATH1061 | Week 9 Tutorial (T5)}
\date{30 April 2013}
\maketitle

\begin{enumerate}
\item
Do graphs with the following descriptions exist? If so, draw one, if not,
explain why.
    \begin{enumerate}
    \item A graph on 4 vertices with vertices of degree 1, 2, 3, 3.
    \item A graph on 5 vertices with vertices of degree 1, 2, 3, 3, 5.
    \item A simple graph on 6 vertices, all with degree 3.
    \item A tree on 7 vertices with no leaves.
    \item A tree on 5 vertices which contains a Hamiltonian path.
    \item A connected graph with 8 vertices and 6 edges.
    %\item A graph with the following incidence matrix:
    %\[
    %    J = \begin{pmatrix}
    %        1 & 0 & 0 & 0 & 0 & 1 \\
    %        0 & 0 & 1 & 0 & 1 & 0 \\
    %        0 & 2 & 0 & 0 & 0 & 0 \\
    %        0 & 0 & 1 & 1 & 0 & 1 \\
    %        1 & 0 & 0 & 0 & 1 & 0 \\
    %    \end{pmatrix}
    %\]
    \end{enumerate}
\item
The complete bipartite graph $K_{1,n}$ is a special type of tree, called a
{\em star}. Draw the star $K_{1,5}$, and write its adjacency matrix and
incidence matrix. Describe the adjacency matrix of $K_{1,n}$.
\item
Recall from lectures that a connected graph has an Euler circuit if and
only if every vertex of has even degree, and that it has an Euler path from
$v$ to $w$ if and only if $v$ and $w$ have odd degree and all other vertices
have even degree.

In a directed graph, the {\em out degree} of a vertex $v$ (denoted $\deg^+(v)$)
is the number of arcs going out of $v$, and the {\em in degree} of $v$
(denoted $\deg^-(v)$) is the number of arcs going into $v$.

Let $G$ be a directed, connected graph (i.e. there is a path from each vertex
to every other vertex that follows the directions of the edges). Make a
statement about when $G$ has an Euler circuit. Make another statement about
when $G$ has an Euler path from $v$ to $w$.
\item
Can the following figures be drawn in a single pen stroke? Explain why/why not.
If not, what's the minimum number of pen lifts required?
\begin{center}
\setlength{\unitlength}{3.5cm}
\begin{picture}(3,1)
\put(0,0){\line(1,0){1}}
\put(0,0.6){\line(1,0){1}}
\put(0,0){\line(0,1){0.6}}
\put(1,0){\line(0,1){0.6}}
\put(0,0){\line(5,3){1}}
\put(1,0){\line(-5,3){1}}
\put(0,0.6){\line(3,2){0.5}}
\put(1,0.6){\line(-3,2){0.5}}
\put(1,0){\line(1,1){0.3}}
\put(1,0.6){\line(1,-1){0.3}}
%\put(0.5,0){\line(5,3){0.5}}
%\put(0,0.3){\line(5,3){0.5}}
%\put(0.5,0){\line(-5,3){0.5}}
%\put(1,0.3){\line(-5,3){0.5}}

\put(2.25,0){\line(1,0){0.5}}
\put(2.125,0.25){\line(1,2){0.25}}
\put(2.875,0.25){\line(-1,2){0.25}}
\put(2.5,0){\line(1,2){0.25}}
\put(2.5,0){\line(-1,2){0.25}}
\put(2.25,0.5){\line(1,0){0.5}}
\put(2.375,0.75){\line(1,0){0.25}}
\put(2.25,0){\line(1,2){0.375}}
\put(2.25,0){\line(-1,2){0.125}}
\put(2.75,0){\line(-1,2){0.375}}
\put(2.75,0){\line(1,2){0.125}}
\put(2.125,0.25){\line(1,0){0.75}}
\end{picture}
\end{center}

%\item
%{\em Challenge Question}:
%One way to represent and store a graph in a computer program is to
%store the adjacency matrix of the graph. (If you're doing MATH1051, you will be
%familiar with writing computer programs in MATLAB that use matrices.)
%
%Fact: If $A$ is the adjacency matrix of a graph $G$, the $(i, j)$ entry in the
%matrix $A^k$ is the number of walks of length $k$ from $v_i$ to $v_j$.
%
%Task: If a computer has the adjacency matrix $A$ of a graph, use the above fact
%to design a process that the computer can use to find the length of the
%shortest path from $v_i$ to $v_j$.

\end{enumerate}

\newpage

{\bf Solutions:}

\begin{enumerate}
\item
    \begin{enumerate}
    \item Not possible: total degree is 9, which is odd.
    \item Possible, but not with simple graphs: An example would be to have
    a vertex of degree 5 with a loop, and a vertex of degree 3 with a loop.
    \item Possible: vertex set $\{1, 2, 3, 4, 5, 6\}$ and edge set
    $\{12, 23, 34, 45, 56, 61, 14, 26, 35\}$.
    \item Not possible: Each vertex would have degree at least 2, for a total
    degree of at least 14, but a tree on 7 vertices needs to have total degree
    12.
    \item Possible: A line, with edge set $\{12, 23, 34, 45\}$.
    \item Not possible: Using 6 edges, we can only connect at most 7 vertices.
%    \item Not possible: The 4th edge (4th column of the matrix) only has one
%    endpoint.
    \end{enumerate}
\item
    (TODO Picture)
    Adjacency matrix $A$ and incidence matrix $J$ below:
    \[
    A = \begin{pmatrix}
        0 & 1 & 1 & 1 & 1 & 1 \\
        1 & 0 & 0 & 0 & 0 & 0 \\
        1 & 0 & 0 & 0 & 0 & 0 \\
        1 & 0 & 0 & 0 & 0 & 0 \\
        1 & 0 & 0 & 0 & 0 & 0 \\
        1 & 0 & 0 & 0 & 0 & 0
    \end{pmatrix}
    \quad
    J = \begin{pmatrix}
        1 & 1 & 1 & 1 & 1 \\
        1 & 0 & 0 & 0 & 0 \\
        0 & 1 & 0 & 0 & 0 \\
        0 & 0 & 1 & 0 & 0 \\
        0 & 0 & 0 & 1 & 0 \\
        0 & 0 & 0 & 0 & 1 \\
    \end{pmatrix}
    \]
\item
    Suppose $G$ is adirected connected graph.
    $G$ has an Eulerian cycle if and only if every vertex $u$ has
    $\deg^+(u) = \deg^-(u)$.

    $G$ has an Eulerian path from $v$ to $w$ if and
    only if vertex $\deg^+(v) = \deg^-(v) + 1$, $\deg^-(w) = \deg^+(w)+1$,
    and every other vertex $u$ has $\deg^+(u) = \deg^-(v)$.
\item
    Note that a figure can be drawn in one pen stroke if and only if a graph
    with that drawing, and vertices where lines meet, has an Eulerian
    cycle/path.

    The first figure has two `vertices' of odd degree (the bottom-left
    has degree 3, and the top-right of the rectangle has degree 5), so there
    is an Eulerian path, so it is possible. Such a drawing would have to start
    at one of those two corners, and finish at the other.

    The second figure has six `vertices' of odd degree (the six `corners' all
    have degree 3), so there is no Eulerian path or cycle, so it is not
    possible.
\end{enumerate}

\end{document}
