\documentclass[a4paper,12pt]{article}
\usepackage[lmargin=2cm,rmargin=2cm,tmargin=2cm,bmargin=2cm]{geometry}
\usepackage{amsmath}
\usepackage{amssymb}

\newcommand{\X}{\mathcal{X}}
\newcommand{\Y}{\mathcal{Y}}
\newcommand{\calZ}{\mathcal{Z}}
\newcommand{\Z}{\mathbb{Z}}
\newcommand{\N}{\mathbb{N}}
\newcommand{\Q}{\mathbb{Q}}
\newcommand{\R}{\mathbb{R}}
\begin{document}

\pagestyle{myheadings}
\markright{MATH1061 | Challenge Questions}

\title{MATH1061 | Challenge Questions}
\date{27 September 2013}
\maketitle

\section{Sets}

\begin{enumerate}
\item For a finite set $X$, show that $|\mathcal{P}(X)| = 2^{|X|}$ (hint: use
induction).
\end{enumerate}

\section{Graphs}

\begin{enumerate}
\item Show that a tree on $n \geq 2$ vertices has at least one leaf
(hint: pick a vertex, and start drawing a path away from it).
\item Repeat the process in the hint to show that a tree on $n \geq 2$ vertices
has at least two leaves.
\item Prove by induction that a tree on $n$ vertices has $n-1$ edges.
(Hint: if you remove a leaf from a tree, the result is also a tree)
\item Knowing that a tree on $n$ vertices has $n-1$ edges, give an alternate
proof that a tree on $n \geq 2$ vertices has at least two leaves.
\end{enumerate}

\section{Relations}

For a relation $\rho$ on a set $A$, we say that $\rho$ is {\em irreflexive} if
for all $x \in A$, it does {\bf not} hold that $x \rho x$.
We say that $\rho$ is {\em asymmetric} if, whenever $x \rho y$, it does not
hold that $y \rho x$.

\begin{enumerate}
\item Let $A = \{1,2,3\}$.
\begin{enumerate}
\item State a relation on $A$ which is both symmetric and antisymmetric.
\item State a relation on $A$ which is neither symmetric nor antisymmetric.
\item State a relation on $A$ which is neither reflexive nor irreflexive.
\item State a relation on $A$ which is neither symmetric nor asymmetric.
\end{enumerate}
\item For any set $A$, find a relation which is both an equivalence relation
and a partial order.
\item We say that a relation is a ``strict partial order'' if it is
irreflexive, asymmetric and transitive.
\begin{enumerate}
\item Let $A$ be a set, and let $\preceq$ be a partial order on $A$.
Define a relation $\prec$ on $A$ by: $a \prec b$ iff $a \preceq b$ and
$a \neq b$. Show that $\prec$ is a strict partial order.
\item
Let $\prec$ be a strict partial order on $A$, and define a relation $\preceq$
on $A$ by: $a \preceq b$ iff $a \prec b$ or $a = b$. Show that $\preceq$ is
a partial order.
\end{enumerate}
\item Let $\rho$ be a relation on a set $A$ which is symmetric and transitive.
Show that $\rho$ is an equivalence relation.
\end{enumerate}

\section{Functions and Cardinality}

In the following questions, our aim is to find a sensible way to talk about the
`size' of an infinite set, by using properties about the size of finite sets.

\begin{enumerate}
\item Let $X = \{1, 2, 3\}$, and $Y = \{1, 2, 3, 4\}$.
\begin{enumerate}
\item Construct a one-to-one function from $X$ to $Y$, and an onto function
from $Y$ to $X$.
\item Explain why there are no surjections from $X$ to $Y$.
\item Explain why there are no injections from $Y$ to $X$.
\end{enumerate}

\item Let $A$ and $B$ be finite sets.
\begin{enumerate}
\item Prove that $|A| \leq |B|$ iff there exists an injective
function from $A$ to $B$.
\item Prove that $|A| \geq |B|$ iff there exists a surjective
function from $A$ to $B$.
\item Prove that $|A| = |B|$ iff there exists a bijective
function from $A$ to $B$.
\end{enumerate}

\item
\begin{enumerate}
\item Let $\iota_\X$ be the identity function on a set $\X$. Prove that
$\iota_\X$ is a bijection.
\item If $f : \X \to \Y$ is a bijection, prove that $f^{-1}$ is a bijection.
\item If $f : \X \to \Y$ and $g : \Y \to \calZ$ are both one-to-one,
prove that $g \circ f$ is one-to-one. (This is a theorem from workbook chapter H.4.)
\item If $f : \X \to \Y$ and $g : \Y \to \calZ$ are both onto functions,
prove that $g \circ f$ is onto. (This is a theorem from workbook chapter H.4.)
\item Define a relation $\sim$ on sets such that for any sets $A$ and $B$,
$A \sim B$ if and only if there exists a bijection from $A$ to $B$. Show that
$\sim$ satisfies the three axioms for being an equivalence relation.
\item Conclude from 2(c) and 3(e)
that if $A$ and $B$ are finite, then $A \sim B$ iff $|A|=|B|$.
\end{enumerate}

\item
For any sets $A$ and $B$ (possibly infinite), we say that $A$ and $B$ have the
same `cardinality' if $A \sim B$, and we write $|A|=|B|$.
Note that `cardinality' can be thought of as `size', but also allowing infinite
sets. Read Chapter H.5 of the workbook before proceeding.
\begin{enumerate}
\item Let $\Z$ be the set of integers and let $\Z_{even}$ be the set of even
integers. Show that $|\Z| = |\Z_{even}|$, i.e. show that they have the
same cardinality, even though $\Z$ contains seemingly `more' elements than
$\Z_{even}$.
\item Let $\Z^+$ be the set of positive integers, also denoted $\N$.
Research Cantor's diagonal argument, which shows that $|\R| \neq |\Z^+|$,
hence $\R$ is uncountable.
\item Show that $\Z$ is countable, i.e. $|\Z| = |\Z^+|$.
\item Show that $\Q$ is countable.
\item Let $(-\frac{\pi}{2}, \frac{\pi}{2}) = \{x \in \R \mid -\frac{\pi}{2}
< x < \frac{\pi}{2}\}$, i.e. the interval of
real numbers between $-\frac{\pi}{2}$ and $\frac{\pi}{2}$. Show that this
interval has the same cardinality as $\R$.
\item Show that the interval $(0, 1) = \{x \in \R \mid 0 < x < 1\}$ has the
same cardinality as $(-\frac{\pi}{2}, \frac{\pi}{2})$.
\item Conclude that $(0, 1)$ is uncountable.
\end{enumerate}

\item Now that we have different `sizes' of infinity, we would like to compare
them, and talk about when one cardinality is `bigger' than another.

\begin{enumerate}
\item
For sets $A$, $B$, define a relation $\preceq$ such that $A \preceq B$ iff
there exists an injection from $A$ to $B$.

Let $A, B, C$ be sets where $A$ has the same cardinality as $B$. Show that
if $A \preceq C$, then $B \preceq C$; and if $C \preceq A$, then $C \preceq B$.
Hence conclude that this relation can be instead thought of as a relation on
cardinalities, instead of a relation on sets.
\item
For sets $A$, $B$ with cardinalities $|A|$ and $|B|$, we say that
$|A| \leq |B|$ if there exists an injection from $A$ to $B$.
(Note that for the size of finite sets, this definition agrees with the result
from 2(a).)
Use the results from 3(a) and 3(c) to conclude that this relation is
reflexive and transitive.
\item The Cantor-Bernstein-Schroeder theorem shows that if there is an injection
from $A$ to $B$, and an injection from $B$ to $A$, then there is a bijection
from $A$ to $B$.
Conclude from this that the relation $\leq$ is a partial order on cardinalities.
\item If $|A| \leq |B|$ and $|A| \neq |B|$, we say that $|A| < |B|$.
Show that $|\Z^+| < |\R|$.
\item Show that $|\Q| < |(0, 1)|$, hence conclude that there are more numbers
between $0$ and $1$ than there are rational numbers in the entire number line.
\item If $A$ is finite and $B$ is infinite, show that $|A| < |B|$.
\item Using the theorem from page 244 of the workbook (known as Cantor's
theorem), show that $|X| < |\mathcal{P}(X)|$ for any set $X$.
Hence, show that there are infinitely many types of infinite cardinalities
(hint: consider $\Z$, $\mathcal{P}(\Z)$, $\mathcal{P}(\mathcal{P}(\Z))$, \dots).
\end{enumerate}
\end{enumerate}

\end{document}
