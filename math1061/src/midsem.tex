\documentclass{beamer}
\usepackage{amsmath}
\usepackage{graphicx}
\usepackage{amssymb}
\newcommand{\n}{\sim}
\newcommand{\floor}[1]{\left\lfloor #1 \right\rfloor}
\newcommand{\ceil}[1]{\left\lceil #1 \right\rceil}
\begin{document}

\begin{frame}
\frametitle{Question 1}
Write negations for each of the following statements:

\begin{enumerate}
\item
"John is 6 feet tall and he weights at least 200 pounds."

\pause
"John is not 6 feet tall, or he weighs less than 200 pounds."

\pause
\item
$-1 < x \leq 4$.

\pause
$x \leq -1$ or $x > 4$. ($\lor$ may be used instead of "or")
\end{enumerate}
\end{frame}

\begin{frame}
\frametitle{Question 2}
Show that the following statements $P$ and $Q$ are logically equivalent:
\[ P : \; \n ( \n p \land q) \land (p \lor q) \land q \]
\[ Q : \; p \land q \]

\pause
\begin{tabular}{cc|c|c}
p & q & P & Q \\
\hline
T & T & T & T \\
T & F & F & F \\
F & F & F & F \\
F & F & F & F
\end{tabular}

or,
\[ P \equiv (q \to p) \land (p \lor q) \land q
     \equiv (q \to p) \land q
     \equiv p \land q = Q \]
\end{frame}

\begin{frame}
\frametitle{Question 3}

Write the negations for each of the following statements.

\begin{enumerate}
\item
"If my car is in the repair shop, then I cannot get to class."

\pause
"My car is in the repair shop and I can get to class."

\pause
\item
"If Sara lives in Athens, then she lives in Greece."

\pause
"Sara lives in Athens, but not in Greece."
\end{enumerate}

\end{frame}

\begin{frame}
\frametitle{Question 4}
Rewrite the following statements in the form "If A then B":
\begin{enumerate}
\item
Pia's birth on U.S. soil is a sufficient condition for her to be a U.S. citizen.

\pause
If (Pia is born on U.S. soil), then (Pia is a U.S. citizen).

\pause
\item
George's attaining age 35 is a necessary condition for his being president of
the U.S.

\pause
If (George is president of the U.S.), then (George has attained age 35).
\end{enumerate}

\end{frame}

\begin{frame}
\frametitle{Question 5}
Is the following argument valid or invalid?
\[
    (p \lor q) \land (p \to r) \land (q \to r) \to r
\]

\pause
Assume that it's invalid. Then,
\begin{itemize}
\pause
\item $\n r$ (by assuming conclusion is false)
\pause
\item $\n p$ (by $p \to r$ and $\n r$)
\pause
\item $\n q$ (by $q \to r$ and $\n r$)
\end{itemize}
\pause
But then, $p \lor q$ cannot be true. Contradiction!

\pause
So, our assumption that the argument invalid is wrong, so the argument is valid.

\end{frame}

\begin{frame}
\frametitle{Question 6}
Write the formal negations of the following statements. Are they true or false?

\begin{enumerate}
\item
For all primes $p$, $p$ is odd.

\pause
"$\exists p \in \mathbb{P} : p$ is even."

The negation is true, $p = 2$ is an example of an even prime. Here,
$\mathbb{P}$ represents the set of all prime numbers.

\pause
\item
There exists a triangle $T$ such that the sum of the angles of $T$ equals 200
degrees.

\pause
Let $\mathbb{T}$ be the set of all triangles.
"$\forall T \in \mathbb{T}$, the sum of the angles of $T \neq 200^\circ$."

The negation is true, the sum of the angles on a triangle is $180^\circ$ for
all triangles.

(Correction 29/8/12: Used more formal mathematical language.)
\end{enumerate}

\end{frame}

\begin{frame}
\frametitle{Question 7}
Rewrite the following statements as quantified conditional statements.
\begin{enumerate}
\item
Squareness is a sufficient condition for rectangularity.

\pause
For all polygons $P$, $P$ is a square $\to$ $P$ is a rectangle.

\pause
\item
A product of two numbers is zero only if one of the numbers is zero.

\pause
\[
    \forall x, y \in \mathbb{R}, \; xy = 0 \to x = 0 \lor y = 0.
\]

(Correction 29/8/12: fixed the solution.)
\end{enumerate}
\end{frame}

\begin{frame}
\frametitle{Question 8(a-b)}
Prove the following statements:
\begin{enumerate}
\item
There exists an even integer n that can be written in two ways as a sum of
two prime numbers.

\pause
Example:
\[
    10 = 5 + 5 = 3 + 7.
\]
\pause
\item
For two integers $r$ and $s$, there exists an integer $k$ such that
$22r + 18s = 2k$.

\pause
For integers $r$ and $s$, $k = 11r + 9s$ is an integer such that
$22r + 18s = 2k$.
\end{enumerate}
\end{frame}
\begin{frame}
\frametitle{Question 8(c-d)}

\begin{enumerate}
\item
The sum of any two rational numbers is rational.

\pause
If $a, b$ are rational, then $a = \frac{p}{q}$, $b = \frac{r}{s}$ for some
$p,q,r,s \in \mathbb{Z}$ where $q, s \neq 0$. \pause Then,
\[
    a + b = \frac{p}{q} + \frac{r}{s} = \frac{ps + qr}{qs},
\]
\pause
$ps+qr, qs \in \mathbb{Z}$, and $qs \neq 0$, so $a+b$ is rational.

\pause
\item
For all integers $a$, $b$, and $c$, if $a|b$ and $b|c$, then $a|c$.

\pause
If $a|b$ and $b|c$, then $b = ma$ and $c = nb$ for some $m, n \in \mathbb{Z}$.
\pause Then, $c = n(ma) = (nm)a$, so $a | c$ (since $nm \in \mathbb{Z}$).
\end{enumerate}
\end{frame}

\begin{frame}
\frametitle{Question 9}
Prove or disprove the following statement: For all integers $a$ and $b$, if
$a|b$ and $b|a$, then $a = b$.

\pause
$1 | -1$ and $-1 | 1$, but $1 \neq -1$, so the statement is false.
\end{frame}

\begin{frame}
\frametitle{Question 10}
For an integer $m$, if $m \equiv 6 \mod 11$, what is $4m \mod 11$?

\pause
\[
    4m \equiv 4 \cdot 6 \equiv 24 \equiv 2 \mod 11.
\]
\end{frame}

\begin{frame}
\frametitle{Question 11}
Prove the following statement: If $m$ is the square of an odd integer, then
\[ m \equiv 1 \mod 8. \]

\pause
If $m$ is the square of an odd integer, then
\[
    m = (2k+1)^2 = 4k^2 + 4k + 1 = 4k(k+1) + 1
\]
for some $k \in \mathbb{Z}$.
\pause
One of $k$ and $k+1$ is even, so $k(k+1)$ is even, and $4k(k+1) = 8n$ for some
$n \in \mathbb{Z}$.
\pause
So, $m = 8n + 1$, therefore $m \equiv 1 \mod 8$.
\end{frame}

\begin{frame}
\frametitle{Question 12}
For an integer $k$, what are $\floor{k}$ and $\floor{k + \frac{1}{2}}$?

\pause
$\floor{k} = \floor{k+\frac{1}{2}} = k$.

\end{frame}

\begin{frame}
\frametitle{Question 13}
Prove or disprove the following statements.
\begin{enumerate}
\item
For all real numbers $x$ and $y$, $\floor{x + y} = \floor{x} + \floor{y}$.

\pause
If $x = y = \frac{1}{2}$,
\[
    \floor{\frac{1}{2} + \frac{1}{2}} = 1
    \neq \floor{\frac{1}{2}} + \floor{\frac{1}{2}} = 0,
\]
so the statement is false.
\pause
\item
For all real numbers $x$ and all integers $m$, $\floor{x+m} = \floor{x} + m$.

\pause
Let $n = \floor{x}$, i.e. $n$ is an integer such that $n \leq x < n + 1$.

\pause
Then, $n + m$ is an integer s.t. $n + m \leq x + m < n + m + 1$.

\pause
Therefore, $\floor{x + m} = n + m = \floor{x} + m$.

\end{enumerate}


\end{frame}

\end{document}
