\documentclass[a4paper,12pt]{article}
\usepackage{amsmath}
\usepackage{amssymb}
\usepackage{amsthm}

\newcommand{\floor}[1]{\left\lfloor #1 \right\rfloor}
\newcommand{\ceil}[1]{\left\lceil #1 \right\rceil}
\begin{document}

\pagestyle{myheadings}
\markright{MATH1061 | Week 4 Tutorial (T5)}

\title{MATH1061 | Week 4 Tutorial (T5)}
\date{19 March 2013}
\maketitle

Prove or disprove the following claims:

\begin{enumerate}
\item There is an integer $n$ such that $n > 3$ and $2^n - 1$ is prime.
\item Every integer is either even or odd.
\item The sum of any two irrational numbers is irrational.
%\item The sum of any two rational numbers is rational.
\item $\forall r, s \in \mathbb{Q}$, if $r < s$, then $\exists t \in \mathbb{Q}$ such that $r < t < s$.
%      (Recall that ``$x \in \mathbb{Q}$'' means that $x$ is a rational number)
\item $\forall a, b, c \in \mathbb{Z}$, if $a \mid c$ and $b \mid c$, then $(a + b) \mid c$.
\item There is an integer $n$ such that $n^2 \equiv 2 \; (\text{mod } 3)$.
\item $\forall x \in \mathbb{R}$, $\floor{x} \leq \ceil{x}$.
\end{enumerate}

\newpage

Hints:

\begin{enumerate}
\item The claim is true | Give an example.
\\
\item The claim is true | Substitute $d = 2$ into the Quotient-Remainder
Theorem (p. 58 workbook); we want to show that every integer satisfies the definition of
`even' or `odd' (p. 43 workbook).
\\
\item The claim is false | Find a counter-example.
\\
\item The claim is true | The proof will start with something like
``Let $r, s \in \mathbb{Q}$ such that $r < s$, then $r = \frac{a}{b},
s = \frac{c}{d}$ where $a, b, c, d$ are \dots''.
Then use $r$ and $s$ to make a new rational number (call it $t$), and show that
$r < t < s$ and $t$ is rational.
\\
\item The claim is false | Find a counter-example.
\\
\item The claim is false | We need to prove that for any integer $n$, that
$n^2 \not\equiv 2$ (mod 3). We prove this by division into cases, using the
fact that for an integer $n$, either $n \equiv 0$,
or $n \equiv 1$, or $n \equiv 2$ (mod 3). For each of these three cases,
work out what $n^2$ is congruent to (mod 3).
\\
\item The claim is true | Apply the definitions of $\floor{x}$ and $\ceil{x}$
(p. 64 workbook), i.e. ``$\floor{x}$ is an integer $n$ such that
\dots, and $\ceil{x}$ is an integer $m$ such that \dots. Then, \dots''.
Don't use the same letter for $\floor{x}$ and $\ceil{x}$!
\end{enumerate}

\newpage

Solutions:

\begin{enumerate}
\item {\em True} | Example: For $n = 5$, $2^5 - 1 = 31$, which is prime.
\item {\em True} | For any integer $n$, by the Quotient-Remainder Theorem with
$d = 2$, $n = 2q + r$ for some integer $r$ with $0 \leq r < 2$. The only
possibilities are $r = 0$, so $n = 2q$ which is even, or $r = 1$, so $n = 2q+1$
which is odd.
\item {\em False} | Counter-example: $\sqrt{2}$ and $-\sqrt{2}$ are irrational,
but $\sqrt{2} + (-\sqrt{2}) = 0$ is rational.
%\item {\em True} | Let $r$ and $s$ be rational numbers, then $r = \frac{a}{b}$
%and $s = \frac{c}{d}$ for some $a, b, c, d \in \mathbb{Z}$ where $b, d \neq 0$.
%Then, $r + s = \frac{ad + bc}{bd}$. Since $ad + bc \in \mathbb{Z}$ and
%$0 \neq bd \in \mathbb{Z}$, so $r + s$ is rational.
\item {\em True} | Let $r, s \in \mathbb{Q}$ such that $r < s$, then $r = \frac{a}{b},
s = \frac{c}{d}$ for some $a, b, c, d \in \mathbb{Z}$ where $b, d \neq 0$. Then,
let $t = \frac{r+s}{2} = \frac{ad + bc}{2bd}$. Now, since $r < s$, then 
$\frac{r + r}{2} < \frac{r + s}{2} < \frac{s + s}{2}$, so $r < t < s$. Also,
$ad + bc, 2bd \in \mathbb{Z}$, and $2bd \neq 0$, so $t$ is rational.
\item {\em False} | Counter-example: $a = 2, b = 3, c = 6$. Observe that
$2 \mid 6$ and $3 \mid 6$, but $(2 + 3) = 5 \nmid 6$.
\item {\em False} | Prove the negation, by division into cases modulo 3:
if $n \equiv 0 \; (\text{mod } 3)$, then $n^2 \equiv 0^2 = 0 \; (\text{mod } 3)$;
if $n \equiv 1 \; (\text{mod } 3)$, then $n^2 \equiv 1^2 = 1 \; (\text{mod } 3)$;
if $n \equiv 2 \; (\text{mod } 3)$, then $n^2 \equiv 2^2 = 4 \equiv 1 \; (\text{mod } 3)$.
In each of these cases, $n^2 \not\equiv 2$ (mod 3).
\item {\em True} | Let $n = \floor{x}$ and $m = \ceil{x}$, then $n \leq x < n + 1$
and $m-1 < x \leq m$. Therefore, $n \leq x \leq m$, so $n \leq m$. Substituting
in the definitions of $n$ and $m$, $\floor{x} \leq \ceil{x}$.
\end{enumerate}

\end{document}
