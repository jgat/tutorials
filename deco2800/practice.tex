\documentclass[a4paper,11pt]{article}

\usepackage[lmargin=2.8cm,rmargin=2.8cm,tmargin=3cm,bmargin=3cm]{geometry}
\usepackage{hyperref}
\input{pyghead}
\begin{document}

\pagestyle{myheadings}
\markright{DECO2800/7280 Tutorial Questions}

\begin{center}
\bf
DECO2800/7280 | Design Computing Studio 2 - Testing \& Evaluation

Tutorial Questions

v1.0 (October 2013)
\end{center}

\begin{enumerate}
\item
For each of the hypothetical scenarios below, describe briefly how you would
identify the source of the problem, and once it is fixed, what measures you
would take to prevent the problem from happening again.

You have the following tools available; include in your
answer which tool(s) you would use, how you would use them, and how the
information they reveal would help you diagnose the problem:
\begin{itemize}
\item
a version management tool (Git)
\item
an issue tracking/ticketing system (GitHub)
\item
a developer documentation wiki (GitHub)
\item
an automated build system (Gradle)
\item
a continuous integration server (Jenkins)
\item
a debugger (Eclipse)
\item
a profiler (VisualVM)
\item
logging (slf4j)
\item
unit testing frameworks (JUnit, Mockito)
\item
integration testing frameworks (JUnit, Mockito)
\item
static analysis tools (Sonar, PMD, FindBugs, etc)
\item
refactorings
\item
user testing methods
\end{itemize}

\begin{enumerate}
\item
You've developed a client/server application, and you've noticed that the
server occasionally crashes. You're able to connect to the server with the
client and reproduce a sequence of actions to crash the server.

\item
You've developed a client/server application, but if you leave the server
application running for too long, it starts using up a lot of the machine's
memory, and eventually must be restarted.

\item
You've implemented an algorithm, along with a unit test for the algorithm.
Yesterday, your algorithm was working, but you ran the test again today to find
it's now failing. Your colleagues have made changes to other parts of the
codebase since then, and you suspect that they forgot to run the tests, and
unknowingly broke your code.

\item
You've noticed that many of the users who have signed up for your arcade system
haven't logged in recently. You've also heard a couple of users say they can't
play any multiplayer games, but you've tested this and you know that it is
possible to play multiplayer games.
\end{enumerate}

\newpage

\item
Suppose we have a class which doesn't have any unit tests, and we wish to write
a set of unit tests to achieve full path coverage (that is, for each method,
every possible execution path is covered by a test). Which one of the following
metrics would make it significantly easier to achieve full path coverage?
\begin{itemize}
\item Low Number of Lines of Code
\item Low Cyclomatic Complexity
\item High Duplication
\item Low Lack of Cohesion Metric (LCOM4)
\end{itemize}

\item
For each of the following hypothetical scenarios related to the arcade system,
identify a design pattern which you would use in the design of the feature,
and identify the classes which would be used in the pattern.

There is no single `correct answer'; correctness is
determined partially by your ability to justify your answer, and demonstrate
knowledge of the design pattern used.
\begin{enumerate}
\item
In the game Pac-Man, there are a number of ghosts chasing Pac-Man.
Each ghost has a different tactic, so that different ghosts will move in
potentially different directions. If users play in different difficulties
(easy/medium/hard), or progress to different levels, the ghosts may employ
different tactics.

\item
In the arcade's matchmaking service, if you wish to play a multiplayer game,
you connect to the service, which will then present you with another user to
play against (and conversely, that user will be presented with you as an
opponent). You and your opponent will then confirm with the service to begin
the game.
The service must manage a set of users who are connected, decide
which ones to match, present the users with their opponents, and check
that each user has confirmed to begin the game.

\item
In a platform game (similar to Super Mario Bros.), players jump between different
platforms to avoid enemies and reach a goal. Users can buy powerups from the
arcade store, which enhance their character's abilities
(e.g. jump higher, run faster, or become immune to enemies).
Powerups last for the duration of one instance of the game,
and can be composed to gain multiple enhancements at the same time.

\item
In the arcade's spectating service, users can choose to broadcast a game to
the public. Whenever the user makes a move in the game, they will notify the
service, which will then notify all spectators so that they can see the move.
\end{enumerate}

\item
Draw a UML Class Diagram of the classes in the {\tt deco2800.arcade.pong}
package. You do not need to include classes outside of this package.
In your diagram:
\begin{itemize}
\item
Distinguish between classes, abstract classes, and interfaces,
\item
Identify inheritance/realisation relationships,
\item
Identify associations, aggregations, and/or compositions between classes,
\item
Identify attributes and methods, including their visibility and types
(note: this may make your diagram large. You may wish to first sketch your
diagram without including this.)
\end{itemize}

\item
Draw a UML State Diagram for the Pong game in the arcade, showing the transitions
between `Ready', `In Progress', and `Game Over'. Clearly identify the start and
end states.

\item
Draw a UML Sequence Diagram for the method
{\tt Pong.handleInput()} in the Pong game.
Be sure to include the different behaviour of the {\tt GameState} subclasses,
as well as any object creation or deletion, loops, branches, etc.
You do not need to show the detail of the methods outside the
{\tt deco2800.arcade.pong} package, including
{\tt incrementAchievement}, {\tt sendNetworkObject}, or any of the GDX methods.

\item
Design a set of functional test cases for the {\tt Pong.endPoint} method to
achieve full branch coverage. For each test, describe the setup/inputs, and
the expected results (including expected changes to the state of the {\tt Pong}
class). State which classes should be mocked in the test design.

You may assume that other methods in the class, such as
{\tt incrementAchievement} and {\tt createScoreUpdate}, are correct, and don't
need testing.

\item
You've invented a new card game, called
{\em 499}, and implemented a version of the game in the DECO2800
arcade system. The game can be played in multiplayer mode against other people
over the network, or in single-player mode against artificial intelligences.

You have personally designed and implemented all the rules for the game, the
AI for single-player mode, and the interface to play the game on the
arcade client (including the controls to play your cards, and the way you
display the game).

You wish to perform user experience testing on this game. Identify two aspects
of this game which you could potentially test. For each one, briefly describe
what aspect of your design you would seek to evaluate, and what observations
you would hope to gather.

\item
The following method uses JDBC to find the number of credits available to a
given username. Identify and describe three potential problems that could occur
in this method, and describe what should be done instead.

\begin{Verbatim}[commandchars=\\\{\}]
\PY{k+kd}{public} \PY{k+kt}{int} \PY{n+nf}{getUserCredits}\PY{o}{(}\PY{n}{String} \PY{n}{username}\PY{o}{)} \PY{k+kd}{throws} \PY{n}{DatabaseException} \PY{o}{\PYZob{}}
    \PY{n}{Connection} \PY{n}{connection} \PY{o}{=} \PY{n}{getConnection}\PY{o}{(}\PY{o}{)}\PY{o}{;}
    \PY{k}{try} \PY{o}{\PYZob{}}
        \PY{n}{Statement} \PY{n}{statement} \PY{o}{=} \PY{n}{connection}\PY{o}{.}\PY{n+na}{createStatement}\PY{o}{(}
            \PY{l+s}{\PYZdq{}SELECT * FROM CREDITS WHERE username=\PYZbs{}\PYZdq{}\PYZdq{}} \PY{o}{+} \PY{n}{username} \PY{o}{+} \PY{l+s}{\PYZdq{}\PYZbs{}\PYZdq{}\PYZdq{}}\PY{o}{)}\PY{o}{;}
        \PY{n}{ResultSet} \PY{n}{resultSet} \PY{o}{=} \PY{n}{statement}\PY{o}{.}\PY{n+na}{executeQuery}\PY{o}{(}\PY{o}{)}\PY{o}{;}
        \PY{k+kt}{int} \PY{n}{result} \PY{o}{=} \PY{n}{findCreditsForUser}\PY{o}{(}\PY{n}{username}\PY{o}{,} \PY{n}{resultSet}\PY{o}{)}\PY{o}{;}
        \PY{n}{resultSet}\PY{o}{.}\PY{n+na}{close}\PY{o}{(}\PY{o}{)}\PY{o}{;}
        \PY{n}{statement}\PY{o}{.}\PY{n+na}{close}\PY{o}{(}\PY{o}{)}\PY{o}{;}
        \PY{k}{return} \PY{n}{result}\PY{o}{;}
    \PY{o}{\PYZcb{}} \PY{k}{catch} \PY{o}{(}\PY{n}{SQLException} \PY{n}{e}\PY{o}{)} \PY{o}{\PYZob{}}
        \PY{k}{throw} \PY{k}{new} \PY{n+nf}{DatabaseException}\PY{o}{(}\PY{l+s}{\PYZdq{}Unable to get credits\PYZdq{}}\PY{o}{,} \PY{n}{e}\PY{o}{)}\PY{o}{;}
    \PY{o}{\PYZcb{}}
\PY{o}{\PYZcb{}}
\end{Verbatim}


\item
Complete the following questions/tasks:
\begin{enumerate}
\item
CSSE2003 Exam 2012, Questions 1, 4, 7, 10, 11, 12, 13, 14.
\item
\url{https://github.com/wbillingsley/tutorial-thekey-mock-public}
\item
\url{https://github.com/wbillingsley/tutorial-anangelwhenyoulook-jpa}
\item
\url{https://github.com/wbillingsley/tutorial-truthwillsetyoufree}
\item
\url{https://github.com/wbillingsley/tutorial-checkmate}
\end{enumerate}

\end{enumerate}

\end{document}
