\documentclass[a4paper,11pt]{article}
\usepackage[lmargin=2.5cm,rmargin=2.5cm,tmargin=2.5cm,bmargin=2.5cm]{geometry}
\usepackage{amsthm}

\newcommand{\floor}[1]{\left\lfloor #1 \right\rfloor}
\usepackage{fancyvrb}
\usepackage{color}
\usepackage[latin1]{inputenc}

\makeatletter
\def\PY@reset{\let\PY@it=\relax \let\PY@bf=\relax%
    \let\PY@ul=\relax \let\PY@tc=\relax%
    \let\PY@bc=\relax \let\PY@ff=\relax}
\def\PY@tok#1{\csname PY@tok@#1\endcsname}
\def\PY@toks#1+{\ifx\relax#1\empty\else%
    \PY@tok{#1}\expandafter\PY@toks\fi}
\def\PY@do#1{\PY@bc{\PY@tc{\PY@ul{%
    \PY@it{\PY@bf{\PY@ff{#1}}}}}}}
\def\PY#1#2{\PY@reset\PY@toks#1+\relax+\PY@do{#2}}

\expandafter\def\csname PY@tok@gd\endcsname{\def\PY@tc##1{\textcolor[rgb]{0.63,0.00,0.00}{##1}}}
\expandafter\def\csname PY@tok@gu\endcsname{\let\PY@bf=\textbf\def\PY@tc##1{\textcolor[rgb]{0.50,0.00,0.50}{##1}}}
\expandafter\def\csname PY@tok@gt\endcsname{\def\PY@tc##1{\textcolor[rgb]{0.00,0.27,0.87}{##1}}}
\expandafter\def\csname PY@tok@gs\endcsname{\let\PY@bf=\textbf}
\expandafter\def\csname PY@tok@gr\endcsname{\def\PY@tc##1{\textcolor[rgb]{1.00,0.00,0.00}{##1}}}
\expandafter\def\csname PY@tok@cm\endcsname{\let\PY@it=\textit\def\PY@tc##1{\textcolor[rgb]{0.25,0.50,0.50}{##1}}}
\expandafter\def\csname PY@tok@vg\endcsname{\def\PY@tc##1{\textcolor[rgb]{0.10,0.09,0.49}{##1}}}
\expandafter\def\csname PY@tok@m\endcsname{\def\PY@tc##1{\textcolor[rgb]{0.40,0.40,0.40}{##1}}}
\expandafter\def\csname PY@tok@mh\endcsname{\def\PY@tc##1{\textcolor[rgb]{0.40,0.40,0.40}{##1}}}
\expandafter\def\csname PY@tok@go\endcsname{\def\PY@tc##1{\textcolor[rgb]{0.53,0.53,0.53}{##1}}}
\expandafter\def\csname PY@tok@ge\endcsname{\let\PY@it=\textit}
\expandafter\def\csname PY@tok@vc\endcsname{\def\PY@tc##1{\textcolor[rgb]{0.10,0.09,0.49}{##1}}}
\expandafter\def\csname PY@tok@il\endcsname{\def\PY@tc##1{\textcolor[rgb]{0.40,0.40,0.40}{##1}}}
\expandafter\def\csname PY@tok@cs\endcsname{\let\PY@it=\textit\def\PY@tc##1{\textcolor[rgb]{0.25,0.50,0.50}{##1}}}
\expandafter\def\csname PY@tok@cp\endcsname{\def\PY@tc##1{\textcolor[rgb]{0.74,0.48,0.00}{##1}}}
\expandafter\def\csname PY@tok@gi\endcsname{\def\PY@tc##1{\textcolor[rgb]{0.00,0.63,0.00}{##1}}}
\expandafter\def\csname PY@tok@gh\endcsname{\let\PY@bf=\textbf\def\PY@tc##1{\textcolor[rgb]{0.00,0.00,0.50}{##1}}}
\expandafter\def\csname PY@tok@ni\endcsname{\let\PY@bf=\textbf\def\PY@tc##1{\textcolor[rgb]{0.60,0.60,0.60}{##1}}}
\expandafter\def\csname PY@tok@nl\endcsname{\def\PY@tc##1{\textcolor[rgb]{0.63,0.63,0.00}{##1}}}
\expandafter\def\csname PY@tok@nn\endcsname{\let\PY@bf=\textbf\def\PY@tc##1{\textcolor[rgb]{0.00,0.00,1.00}{##1}}}
\expandafter\def\csname PY@tok@no\endcsname{\def\PY@tc##1{\textcolor[rgb]{0.53,0.00,0.00}{##1}}}
\expandafter\def\csname PY@tok@na\endcsname{\def\PY@tc##1{\textcolor[rgb]{0.49,0.56,0.16}{##1}}}
\expandafter\def\csname PY@tok@nb\endcsname{\def\PY@tc##1{\textcolor[rgb]{0.00,0.50,0.00}{##1}}}
\expandafter\def\csname PY@tok@nc\endcsname{\let\PY@bf=\textbf\def\PY@tc##1{\textcolor[rgb]{0.00,0.00,1.00}{##1}}}
\expandafter\def\csname PY@tok@nd\endcsname{\def\PY@tc##1{\textcolor[rgb]{0.67,0.13,1.00}{##1}}}
\expandafter\def\csname PY@tok@ne\endcsname{\let\PY@bf=\textbf\def\PY@tc##1{\textcolor[rgb]{0.82,0.25,0.23}{##1}}}
\expandafter\def\csname PY@tok@nf\endcsname{\def\PY@tc##1{\textcolor[rgb]{0.00,0.00,1.00}{##1}}}
\expandafter\def\csname PY@tok@si\endcsname{\let\PY@bf=\textbf\def\PY@tc##1{\textcolor[rgb]{0.73,0.40,0.53}{##1}}}
\expandafter\def\csname PY@tok@s2\endcsname{\def\PY@tc##1{\textcolor[rgb]{0.73,0.13,0.13}{##1}}}
\expandafter\def\csname PY@tok@vi\endcsname{\def\PY@tc##1{\textcolor[rgb]{0.10,0.09,0.49}{##1}}}
\expandafter\def\csname PY@tok@nt\endcsname{\let\PY@bf=\textbf\def\PY@tc##1{\textcolor[rgb]{0.00,0.50,0.00}{##1}}}
\expandafter\def\csname PY@tok@nv\endcsname{\def\PY@tc##1{\textcolor[rgb]{0.10,0.09,0.49}{##1}}}
\expandafter\def\csname PY@tok@s1\endcsname{\def\PY@tc##1{\textcolor[rgb]{0.73,0.13,0.13}{##1}}}
\expandafter\def\csname PY@tok@sh\endcsname{\def\PY@tc##1{\textcolor[rgb]{0.73,0.13,0.13}{##1}}}
\expandafter\def\csname PY@tok@sc\endcsname{\def\PY@tc##1{\textcolor[rgb]{0.73,0.13,0.13}{##1}}}
\expandafter\def\csname PY@tok@sx\endcsname{\def\PY@tc##1{\textcolor[rgb]{0.00,0.50,0.00}{##1}}}
\expandafter\def\csname PY@tok@bp\endcsname{\def\PY@tc##1{\textcolor[rgb]{0.00,0.50,0.00}{##1}}}
\expandafter\def\csname PY@tok@c1\endcsname{\let\PY@it=\textit\def\PY@tc##1{\textcolor[rgb]{0.25,0.50,0.50}{##1}}}
\expandafter\def\csname PY@tok@kc\endcsname{\let\PY@bf=\textbf\def\PY@tc##1{\textcolor[rgb]{0.00,0.50,0.00}{##1}}}
\expandafter\def\csname PY@tok@c\endcsname{\let\PY@it=\textit\def\PY@tc##1{\textcolor[rgb]{0.25,0.50,0.50}{##1}}}
\expandafter\def\csname PY@tok@mf\endcsname{\def\PY@tc##1{\textcolor[rgb]{0.40,0.40,0.40}{##1}}}
\expandafter\def\csname PY@tok@err\endcsname{\def\PY@bc##1{\setlength{\fboxsep}{0pt}\fcolorbox[rgb]{1.00,0.00,0.00}{1,1,1}{\strut ##1}}}
\expandafter\def\csname PY@tok@kd\endcsname{\let\PY@bf=\textbf\def\PY@tc##1{\textcolor[rgb]{0.00,0.50,0.00}{##1}}}
\expandafter\def\csname PY@tok@ss\endcsname{\def\PY@tc##1{\textcolor[rgb]{0.10,0.09,0.49}{##1}}}
\expandafter\def\csname PY@tok@sr\endcsname{\def\PY@tc##1{\textcolor[rgb]{0.73,0.40,0.53}{##1}}}
\expandafter\def\csname PY@tok@mo\endcsname{\def\PY@tc##1{\textcolor[rgb]{0.40,0.40,0.40}{##1}}}
\expandafter\def\csname PY@tok@kn\endcsname{\let\PY@bf=\textbf\def\PY@tc##1{\textcolor[rgb]{0.00,0.50,0.00}{##1}}}
\expandafter\def\csname PY@tok@mi\endcsname{\def\PY@tc##1{\textcolor[rgb]{0.40,0.40,0.40}{##1}}}
\expandafter\def\csname PY@tok@gp\endcsname{\let\PY@bf=\textbf\def\PY@tc##1{\textcolor[rgb]{0.00,0.00,0.50}{##1}}}
\expandafter\def\csname PY@tok@o\endcsname{\def\PY@tc##1{\textcolor[rgb]{0.40,0.40,0.40}{##1}}}
\expandafter\def\csname PY@tok@kr\endcsname{\let\PY@bf=\textbf\def\PY@tc##1{\textcolor[rgb]{0.00,0.50,0.00}{##1}}}
\expandafter\def\csname PY@tok@s\endcsname{\def\PY@tc##1{\textcolor[rgb]{0.73,0.13,0.13}{##1}}}
\expandafter\def\csname PY@tok@kp\endcsname{\def\PY@tc##1{\textcolor[rgb]{0.00,0.50,0.00}{##1}}}
\expandafter\def\csname PY@tok@w\endcsname{\def\PY@tc##1{\textcolor[rgb]{0.73,0.73,0.73}{##1}}}
\expandafter\def\csname PY@tok@kt\endcsname{\def\PY@tc##1{\textcolor[rgb]{0.69,0.00,0.25}{##1}}}
\expandafter\def\csname PY@tok@ow\endcsname{\let\PY@bf=\textbf\def\PY@tc##1{\textcolor[rgb]{0.67,0.13,1.00}{##1}}}
\expandafter\def\csname PY@tok@sb\endcsname{\def\PY@tc##1{\textcolor[rgb]{0.73,0.13,0.13}{##1}}}
\expandafter\def\csname PY@tok@k\endcsname{\let\PY@bf=\textbf\def\PY@tc##1{\textcolor[rgb]{0.00,0.50,0.00}{##1}}}
\expandafter\def\csname PY@tok@se\endcsname{\let\PY@bf=\textbf\def\PY@tc##1{\textcolor[rgb]{0.73,0.40,0.13}{##1}}}
\expandafter\def\csname PY@tok@sd\endcsname{\let\PY@it=\textit\def\PY@tc##1{\textcolor[rgb]{0.73,0.13,0.13}{##1}}}

\def\PYZbs{\char`\\}
\def\PYZus{\char`\_}
\def\PYZob{\char`\{}
\def\PYZcb{\char`\}}
\def\PYZca{\char`\^}
\def\PYZam{\char`\&}
\def\PYZlt{\char`\<}
\def\PYZgt{\char`\>}
\def\PYZsh{\char`\#}
\def\PYZpc{\char`\%}
\def\PYZdl{\char`\$}
\def\PYZhy{\char`\-}
\def\PYZsq{\char`\'}
\def\PYZdq{\char`\"}
\def\PYZti{\char`\~}
% for compatibility with earlier versions
\def\PYZat{@}
\def\PYZlb{[}
\def\PYZrb{]}
\makeatother


\begin{document}

\pagestyle{myheadings}
\markright{CSSE1001 Recursion Challenge Questions}

\begin{enumerate}
\item[Q1.]
The following incomplete recursive function sorts a list of numbers. Write the
missing lines of code (including any recursive call).

\begin{Verbatim}[commandchars=\\\{\},numbers=left,firstnumber=1,stepnumber=1,codes={\catcode`\$=3\catcode`\^=7\catcode`\_=8}]
\PY{k}{def} \PY{n+nf}{sort}\PY{p}{(}\PY{n}{xs}\PY{p}{)}\PY{p}{:}
    \PY{l+s+sd}{\PYZdq{}\PYZdq{}\PYZdq{} Sorts a list of numbers.}
\PY{l+s+sd}{    e.g. sort([3,7,2,4,3]) \PYZhy{}\PYZgt{} [7,4,3,3,2] \PYZdq{}\PYZdq{}\PYZdq{}}
    \PY{k}{if} \PY{o+ow}{not} \PY{n}{xs}\PY{p}{:}
        \PY{k}{return} \PY{o}{.}\PY{o}{.}\PY{o}{.}
    \PY{n}{a} \PY{o}{=} \PY{n+nb}{min}\PY{p}{(}\PY{n}{xs}\PY{p}{)}
    \PY{n}{i} \PY{o}{=} \PY{n}{xs}\PY{o}{.}\PY{n}{find}\PY{p}{(}\PY{n}{a}\PY{p}{)}
    \PY{k}{return} \PY{o}{.}\PY{o}{.}\PY{o}{.}       \PY{c}{\PYZsh{} Hint: take slices of{\tt xs} using i}
\end{Verbatim}


\item[Q2.]
In lectures, we wrote a program which gives instructions to solve the Tower
of Hanoi problem.
The solution described in lectures is to move the first $n-1$ disks from $A$ to
$B$, then move the largest disk from $A$ to $C$, then the $n-1$ disks from $B$
to $C$.

A variant of the problem is to have 3 towers $A$, $B$ and $C$
as before, but you cannot move a piece from $A$ to $C$ or $C$ to $A$.
Find the solution to this problem, then
write a function that takes a number $n$ and prints the instructions to move
the tower from $A$ to $C$, without moving any pieces directly $A \to C$ or
$C \to A$.

Run your function on $n=3$ and verify the output by following the instructions.

\item[Q3.]
Consider the following {\tt sum1} function, which attempts to compute the sum
of a list of numbers. Give an example of a list {\tt xs} where this function
will not work, and explain why it doesn't work.

\begin{Verbatim}[commandchars=\\\{\},numbers=left,firstnumber=1,stepnumber=1,codes={\catcode`\$=3\catcode`\^=7\catcode`\_=8}]
\PY{k}{def} \PY{n+nf}{sum1}\PY{p}{(}\PY{n}{xs}\PY{p}{)}\PY{p}{:}
    \PY{n}{n} \PY{o}{=} \PY{n+nb}{len}\PY{p}{(}\PY{n}{xs}\PY{p}{)}
    \PY{k}{if} \PY{n}{n} \PY{o}{==} \PY{l+m+mi}{0}\PY{p}{:}
        \PY{k}{return} \PY{l+m+mi}{0}
    \PY{k}{else}\PY{p}{:}
        \PY{n}{m} \PY{o}{=} \PY{n}{n} \PY{o}{/} \PY{l+m+mi}{2}  \PY{c}{\PYZsh{} $m = \floor{\frac{n}{2}}$}
        \PY{k}{return} \PY{n}{sum1}\PY{p}{(}\PY{n}{xs}\PY{p}{[}\PY{p}{:}\PY{n}{m}\PY{p}{]}\PY{p}{)} \PY{o}{+} \PY{n}{sum1}\PY{p}{(}\PY{n}{xs}\PY{p}{[}\PY{n}{m}\PY{p}{:}\PY{p}{]}\PY{p}{)}
\end{Verbatim}


\item[Q4.]
Next, consider the following {\tt sum2} function, which is a modification of the
above.
Compute by hand the sum of the list {\tt [3,4,5,6,7]}. In your working,
indicate the recursive calls that are made.

Then, use induction to prove that this function will always return the correct sum.
(Hint: {\tt xs[:m]} is a list of length $m$, and {\tt xs[m:]} is a list of length $n-m$.)

Given that we know {\tt sum1} doesn't work, explain why
we could not use the same technique to prove {\tt sum1} is correct.
(i.e. what part of the proof for {\tt sum2} requires that $n=1$ is a base case?)

\begin{Verbatim}[commandchars=\\\{\},numbers=left,firstnumber=1,stepnumber=1,codes={\catcode`\$=3\catcode`\^=7\catcode`\_=8}]
\PY{k}{def} \PY{n+nf}{sum2}\PY{p}{(}\PY{n}{xs}\PY{p}{)}\PY{p}{:}
    \PY{n}{n} \PY{o}{=} \PY{n+nb}{len}\PY{p}{(}\PY{n}{xs}\PY{p}{)}
    \PY{k}{if} \PY{n}{n} \PY{o}{==} \PY{l+m+mi}{0}\PY{p}{:}
        \PY{k}{return} \PY{l+m+mi}{0}
    \PY{k}{elif} \PY{n}{n} \PY{o}{==} \PY{l+m+mi}{1}\PY{p}{:}
        \PY{k}{return} \PY{n}{xs}\PY{p}{[}\PY{l+m+mi}{0}\PY{p}{]}
    \PY{k}{else}\PY{p}{:}
        \PY{n}{m} \PY{o}{=} \PY{n}{n} \PY{o}{/} \PY{l+m+mi}{2}
        \PY{k}{return} \PY{n}{sum2}\PY{p}{(}\PY{n}{xs}\PY{p}{[}\PY{p}{:}\PY{n}{m}\PY{p}{]}\PY{p}{)} \PY{o}{+} \PY{n}{sum2}\PY{p}{(}\PY{n}{xs}\PY{p}{[}\PY{n}{m}\PY{p}{:}\PY{p}{]}\PY{p}{)}
\end{Verbatim}


\end{enumerate}

\newpage

\subsection*{Solutions}

\begin{enumerate}
\item[Q1.]
A sample {\tt sort} function is as follows.
\begin{Verbatim}[commandchars=\\\{\},numbers=left,firstnumber=1,stepnumber=1,codes={\catcode`\$=3\catcode`\^=7\catcode`\_=8}]
\PY{k}{def} \PY{n+nf}{sort}\PY{p}{(}\PY{n}{xs}\PY{p}{)}\PY{p}{:}
    \PY{l+s+sd}{\PYZdq{}\PYZdq{}\PYZdq{} Sorts a list of numbers.}
\PY{l+s+sd}{    e.g. sort([3,7,2,4,3]) \PYZhy{}\PYZgt{} [7,4,3,3,2] \PYZdq{}\PYZdq{}\PYZdq{}}
    \PY{k}{if} \PY{o+ow}{not} \PY{n}{xs}\PY{p}{:}
        \PY{k}{return} \PY{n}{xs}
    \PY{n}{a} \PY{o}{=} \PY{n+nb}{min}\PY{p}{(}\PY{n}{xs}\PY{p}{)}
    \PY{n}{i} \PY{o}{=} \PY{n}{xs}\PY{o}{.}\PY{n}{find}\PY{p}{(}\PY{n}{a}\PY{p}{)}
    \PY{k}{return} \PY{n}{sort}\PY{p}{(}\PY{n}{xs}\PY{p}{[}\PY{p}{:}\PY{n}{i}\PY{p}{]} \PY{o}{+} \PY{n}{xs}\PY{p}{[}\PY{n}{i}\PY{o}{+}\PY{l+m+mi}{1}\PY{p}{:}\PY{p}{]}\PY{p}{)} \PY{o}{+} \PY{p}{[}\PY{n}{a}\PY{p}{]}
\end{Verbatim}


\item[Q2.]
Solution: move the first $n-1$ disks to $C$, then move
the $n^{\rm th}$ to $B$, then move the first $n-1$ disks back to $A$, then move the
$n^{\rm th}$ to $C$, then move the first $n-1$ back to $C$.
\begin{Verbatim}[commandchars=\\\{\},numbers=left,firstnumber=1,stepnumber=1,codes={\catcode`\$=3\catcode`\^=7\catcode`\_=8}]
\PY{k}{def} \PY{n+nf}{hanoi}\PY{p}{(}\PY{n}{n}\PY{p}{)}\PY{p}{:}
    \PY{n}{\PYZus{}hanoi\PYZus{}helper}\PY{p}{(}\PY{n}{n}\PY{p}{,} \PY{l+s}{\PYZsq{}}\PY{l+s}{A}\PY{l+s}{\PYZsq{}}\PY{p}{,} \PY{l+s}{\PYZsq{}}\PY{l+s}{C}\PY{l+s}{\PYZsq{}}\PY{p}{)}

\PY{k}{def} \PY{n+nf}{\PYZus{}hanoi\PYZus{}helper}\PY{p}{(}\PY{n}{n}\PY{p}{,} \PY{n}{A}\PY{p}{,} \PY{n}{C}\PY{p}{)}\PY{p}{:}
    \PY{k}{if} \PY{n}{n} \PY{o}{==} \PY{l+m+mi}{0}\PY{p}{:}
        \PY{k}{return}
    \PY{n}{\PYZus{}hanoi\PYZus{}helper}\PY{p}{(}\PY{n}{n}\PY{o}{\PYZhy{}}\PY{l+m+mi}{1}\PY{p}{,} \PY{n}{A}\PY{p}{,} \PY{n}{C}\PY{p}{)}
    \PY{k}{print} \PY{n}{A}\PY{p}{,} \PY{l+s}{\PYZdq{}}\PY{l+s}{to B}\PY{l+s}{\PYZdq{}}
    \PY{n}{\PYZus{}hanoi\PYZus{}helper}\PY{p}{(}\PY{n}{n}\PY{o}{\PYZhy{}}\PY{l+m+mi}{1}\PY{p}{,} \PY{n}{C}\PY{p}{,} \PY{n}{A}\PY{p}{)}
    \PY{k}{print} \PY{l+s}{\PYZdq{}}\PY{l+s}{B to}\PY{l+s}{\PYZdq{}}\PY{p}{,} \PY{n}{C}
    \PY{n}{\PYZus{}hanoi\PYZus{}helper}\PY{p}{(}\PY{n}{n}\PY{o}{\PYZhy{}}\PY{l+m+mi}{1}\PY{p}{,} \PY{n}{A}\PY{p}{,} \PY{n}{C}\PY{p}{)}
\end{Verbatim}


\item[Q3.]
Consider the input {\tt xs = [1]}. This will set $n = 1$, go into the {\tt else}
statement, and set $m = 1 / 2 = 0$.
Then, the function will return {\tt sum([]) + sum([1])}. The second of these
calls is the same as the original call, so we will have an infinite recursion.

\item[Q4.]
\begin{itemize}
\item[]
{\tt sum2([3,4,5,6,7]) = sum2([3,4]) + sum2([5,6,7])} $= \dots = 25$
\begin{itemize}
\item[]
{\tt sum2([3,4]) = sum2([3]) + sum2([4])} $= \dots = 7$
\begin{itemize}
\item[]
{\tt sum2([3]) = 3}
\item[]
{\tt sum2([4]) = 4}
\end{itemize}
\item[]
{\tt sum2([5,6,7]) = sum2([5]) + sum2([6,7])} $= \dots = 18$
\begin{itemize}
\item[]
{\tt sum2([5]) = 5}
\item[]
{\tt sum2([6,7]) = sum2([6]) + sum2([7])} $= \dots = 13$
\begin{itemize}
\item[]
{\tt sum2([6]) = 6}
\item[]
{\tt sum2([7]) = 7}
\end{itemize}
\end{itemize}
\end{itemize}
\end{itemize}

{\bf Claim:} The {\tt sum2} function returns the sum of the list {\tt xs}.

Base cases: if $n = 0$, then the list is empty, so the sum is 0 (by definition
of the empty sum). If $n = 1$, then the sum of the list is the value of the
single element.

Inductive hypothesis: For some $n \geq 2$, suppose that the {\tt sum2} function
returns the sum of any list of length $k$, for any $0 \leq k < n$.

Inductive step: If {\tt xs} has length $n \geq 2$, then
$m = \floor{\frac{n}{2}}$ satisfies $1 < m < n$. Then, the sublists of length
$m < n$ and $n-m < n$ will return $\sum_{i=0}^{m-1} {\tt xs[i]}$ and
$\sum_{i=m}^{n-1} {\tt xs[i]}$ respectively, then the function will return
$\sum_{i=0}^{n-1} {\tt xs[i]}$, the correct sum. \qed

If we tried the same proof for {\tt sum1}, it would require the inductive step
to hold for $n \geq 1$ (instead of $n \geq 2$). However, if $n \geq 1$, it is
not guaranteed that $n-m < n$ (since it is possible that $m=0$), hence we
cannot apply the inductive hypothesis to the list {\tt xs[m:]}.
\end{enumerate}

\end{document}
